\documentclass[t]{beamer}
\usepackage[]{graphicx}
\usepackage[table]{xcolor}
\makeatletter
\def\maxwidth{%
  \ifdim\Gin@nat@width>\linewidth
    \linewidth
  \else
    \Gin@nat@width
  \fi
}
\makeatother

\definecolor{fgcolor}{rgb}{0.345, 0.345, 0.345}
\newcommand{\hlnum}[1]{\textcolor[rgb]{0.686,0.059,0.569}{#1}}%
\newcommand{\hlsng}[1]{\textcolor[rgb]{0.192,0.494,0.8}{#1}}%
\newcommand{\hlcom}[1]{\textcolor[rgb]{0.678,0.584,0.686}{\textit{#1}}}%
\newcommand{\hlopt}[1]{\textcolor[rgb]{0,0,0}{#1}}%
\newcommand{\hldef}[1]{\textcolor[rgb]{0.345,0.345,0.345}{#1}}%
\newcommand{\hlkwa}[1]{\textcolor[rgb]{0.161,0.373,0.58}{\textbf{#1}}}%
\newcommand{\hlkwb}[1]{\textcolor[rgb]{0.69,0.353,0.396}{#1}}%
\newcommand{\hlkwc}[1]{\textcolor[rgb]{0.333,0.667,0.333}{#1}}%
\newcommand{\hlkwd}[1]{\textcolor[rgb]{0.737,0.353,0.396}{\textbf{#1}}}%
\let\hlipl\hlkwb

\usepackage{framed}
\makeatletter
\newenvironment{kframe}{%
 \def\at@end@of@kframe{}%
 \ifinner\ifhmode%
  \def\at@end@of@kframe{\end{minipage}}%
  \begin{minipage}{\columnwidth}%
 \fi\fi%
 \def\FrameCommand##1{\hskip\@totalleftmargin \hskip-\fboxsep
 \colorbox{shadecolor}{##1}\hskip-\fboxsep
     \hskip-\linewidth \hskip-\@totalleftmargin \hskip\columnwidth}%
 \MakeFramed {\advance\hsize-\width
   \@totalleftmargin\z@ \linewidth\hsize
   \@setminipage}}%
 {\par\unskip\endMakeFramed%
 \at@end@of@kframe}
\makeatother

\definecolor{shadecolor}{rgb}{.97, .97, .97}
\newenvironment{knitrout}{}{} 

\usepackage{booktabs, textpos, multirow, array, amsmath, amsthm, Sweave}

\usetheme{Dresden}
\setbeamertemplate{headline}{
  \leavevmode%
  \hbox{%
    \begin{beamercolorbox}[wd=\paperwidth,ht=2.5ex,dp=1ex,left]{section in head/foot}
      \hspace*{1em}\insertsection
    \end{beamercolorbox}%
  }
}
\addtobeamertemplate{frametitle}{}{%
    \begin{textblock*}{1cm}(1.11\textwidth,1.105\textheight)
        \color{white}\tiny\insertframenumber{} / \inserttotalframenumber
    \end{textblock*}
}

% --- Presentation information ---
\title{Replication: Factor Momentum}
\subtitle{Based on Arnott, Kalesnik, \& Linnainmaa (2023)}
\author{Caterina Piancentini, Farkas Tallos, Giulio Iepure, Tomas Samaj}
\institute[WU Vienna]{ZZ ILab \\ WU Vienna}
\date{Term 2025/2026 -- ZZ QFin ILab Meeting 2}

\begin{document}

% --- Title Frame ---
{
\begin{frame}[plain]
    \titlepage
\end{frame}
}

% --- Table of Contents ---
\begin{frame}{Table of Contents}
    \tableofcontents
\end{frame}

% ============================
% SECTION 1: Papers and Context
% ============================
\section{Papers and Context}

\begin{frame}{Arnott, Kalesnik \& Linnainmaa (2023) -- \textit{Factor Momentum}}
\textbf{Main Idea:}
\begin{itemize}
    \item Extends momentum research to factor portfolios—showing that factor returns themselves exhibit momentum.
    \item Finds that factor momentum \textbf{subsumes} industry momentum.
    \item Uses principal component analysis to identify systematic sources of momentum.
\end{itemize}

\textbf{Key Insight:}
\begin{itemize}
    \item Momentum is strongest in high-eigenvalue factors explaining most of cross-sectional returns.
    \item Momentum arises from systematic components, not just stock-level trends.
\end{itemize}

\textbf{Relevance:} Our replication reproduces Appendix plots comparing factor and industry momentum.
\end{frame}

\begin{frame}{Ehsani \& Linnainmaa (2022) -- \textit{Factor Momentum and the Momentum Factor}}
\textbf{Contribution:}
\begin{itemize}
    \item Shows that momentum in stock returns stems from momentum in factor returns.
    \item Factors show strong autocorrelation: winners stay winners, losers stay losers.
\end{itemize}

\textbf{Interpretation:}
\begin{itemize}
    \item Momentum reflects timing of factor exposures, not a separate risk factor.
    \item Complements Arnott et al. (2023) by providing theoretical grounding.
\end{itemize}

\textbf{Link:} Our replication validates these findings using Fama–French and JKP datasets.
\end{frame}

\begin{frame}{AQR Alternative Trends UCITS Fund (2025) -- \textit{Practical Application}}
\textbf{Context:}
\begin{itemize}
    \item AQR’s trend-following fund applies cross-asset and factor momentum.
    \item Combines price-based and fundamental trend signals across global assets.
\end{itemize}

\textbf{Performance (Q1 2025 Report):}
\begin{itemize}
    \item Annualized return: 11.9\%; Sharpe ratio: 0.76.
    \item Low equity correlation (-0.20) and positive macro exposure.
\end{itemize}
\end{frame}

% ============================
% SECTION 2: Data
% ============================
\section{Data}

\begin{frame}{Outline}
    \tableofcontents[currentsection, hideallsubsections]
\end{frame}

\begin{frame}{Data Description}
\textbf{Datasets Used:}
\begin{itemize}
    \item \textbf{Fama–French 17 Industry Portfolios:} Monthly excess returns (July 1963–Dec 2020).
    \item \textbf{JKP Factors:} Cross-sectional factor returns (value, profitability, investment, quality, risk).
    \item \textbf{Thematic Factors:} Growth, leverage, volatility, and other economic themes.
\end{itemize}

\textbf{Data Processing:}
\begin{itemize}
    \item Cleaned and aligned to a common monthly sample.
    \item Standardized names and applied readable labels.
\end{itemize}

\textbf{Final Sample:} July 1963 – December 2020 (U.S. market).
\end{frame}

% ============================
% SECTION 3: Replication Methodology
% ============================
\section{Replication Methodology}

\begin{frame}{Outline}
    \tableofcontents[currentsection, hideallsubsections]
\end{frame}

\begin{frame}{Replication Methodology}
\textbf{Objective:} Replicate Arnott et al. (2023) Appendix figure comparing \textbf{Factor} vs. \textbf{Industry Momentum}.

\textbf{Strategy:}
\begin{enumerate}
    \item For each month:
    \begin{itemize}
        \item Rank all assets by previous month’s return.
        \item Go \textbf{long} the top half (winners), \textbf{short} the bottom half (losers).
    \end{itemize}
    \item Apply to Fama–French 17 industries and selected JKP factors.
    \item Equal-weight portfolios; rebalance monthly.
    \item Compute 1-month long–short return.
\end{enumerate}
\end{frame}

% ============================
% SECTION 4: Replication Results
% ============================
\section{Replication Results}

\begin{frame}{Outline}
    \tableofcontents[currentsection, hideallsubsections]
\end{frame}

\begin{frame}
\begin{knitrout}
\definecolor{shadecolor}{rgb}{0.969, 0.969, 0.969}\color{fgcolor}
\begin{figure}
{\centering \includegraphics[width=\maxwidth]{figure/cum_ret_plot-1} }
\caption[Cumulative Performance (Log Scale), July 1963 - Dec 2020]{Cumulative Performance (Log Scale), July 1963 - Dec 2020}
\label{fig:cum_ret_plot}
\end{figure}
\end{knitrout}
\end{frame}

\begin{frame}{Arnott et al. (2023) Plot}
\begin{figure}[ht]
    \centering
    \includegraphics[width=0.92\linewidth]{Actual_plot.png}
    \caption{Industry vs. Factor vs. Industry-Neutral Factor Momentum (1965–2020).}
\end{figure}
\footnotesize
\textbf{Source:} Arnott, Kalesnik, Linnainmaa (2023), \textit{RFS}, 36(8), 3034–3070.
\end{frame}

% ============================
% SECTION 5: Factor Correlation Analysis
% ============================
\section{Factor Correlation Analysis}

\begin{frame}{Outline}
    \tableofcontents[currentsection, hideallsubsections]
\end{frame}

\begin{frame}{Factor Correlation Heatmap (Part 1)}
\textbf{Analysis:}
\begin{itemize}
    \item Correlation calculated among selected JKP factors.
    \item Helps identify relationships between factors being timed.
    \item \textbf{High correlations (dark green):}
    \begin{itemize}
        \item Strong among \textbf{profitability factors} — Gross Profitability, ROE, Profit Margin.
        \item \textbf{Investment factors} (Asset Growth, CAPX Growth) also highly correlated.
        \item Value-style factors moderately correlated with profitability.
    \end{itemize}
    \item Fundamental factors move together, reinforcing systematic momentum.
\end{itemize}
\end{frame}

\begin{frame}{Factor Correlation Heatmap (Part 2)}
\textbf{Analysis (continued):}
\begin{itemize}
    \item \textbf{Low/negative correlations (brown areas):}
    \begin{itemize}
        \item Distress and volatility proxies (Ohlson O-Score, Altman Z-Score, Residual Variance) have weak or negative links with profitability/value.
        \item Size (SMB) and momentum variables are largely orthogonal.
    \end{itemize}
    \item Indicates that some factors add \textbf{diversification} rather than reinforcement.
    \item Overall: Factor momentum mainly comes from \textbf{clusters of correlated, fundamental drivers}.
\end{itemize}
\end{frame}

\begin{frame}
\begin{knitrout}
\definecolor{shadecolor}{rgb}{0.969, 0.969, 0.969}\color{fgcolor}
\begin{figure}
{\centering \includegraphics[width=\maxwidth]{figure/corr_heatmap1-1} }
\caption[Upper Triangular Correlation Heatmap of Selected JKP Factors]{Upper Triangular Correlation Heatmap of Selected JKP Factors}
\label{fig:corr_heatmap1}
\end{figure}
\end{knitrout}
\end{frame}

% ============================
% SECTION 6: Findings and Conclusion
% ============================
\section{Findings and Conclusion}

\begin{frame}{Outline}
    \tableofcontents[currentsection, hideallsubsections]
\end{frame}

\begin{frame}{Key Findings \& Conclusion}
\begin{itemize}
    \item \textbf{Replication Findings:}
    \begin{itemize}
        \item Both industry and factor momentum strategies yield positive returns.
        \item Factor momentum is stronger and more persistent—consistent with Arnott et al. (2023).
    \end{itemize}
    \item \textbf{Interpretation:}
    \begin{itemize}
        \item Factor momentum \textit{subsumes} industry momentum.
        \item Correlation analysis supports clusters of fundamental drivers.
    \end{itemize}
    \item \textbf{Conclusion:}
    \begin{itemize}
        \item Results confirm that short-term momentum is primarily driven by systematic factor dynamics.
    \end{itemize}
\end{itemize}
\end{frame}

\begin{frame}[plain]
    \centering
    \vspace{2cm}
    {\Huge \textbf{Thank You}} \\[0.5cm]
    \Large Questions or comments? \\[1cm]
    \normalsize
    \textit{Replication of Arnott, Kalesnik, \& Linnainmaa (2023)} \\
    WU Vienna – ZZ QFin Lab 2025/26 
\end{frame}

\end{document}